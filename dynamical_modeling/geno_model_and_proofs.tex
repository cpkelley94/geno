% --------------------------------------------------------------
% PREAMBLE
% --------------------------------------------------------------

\documentclass[12pt]{article}

\usepackage[margin=1in]{geometry}
\usepackage{amsmath,amsthm,amssymb}
\usepackage{mathtools}
\usepackage[version=4]{mhchem}
\usepackage{enumitem}

\usepackage[T1]{fontenc}
\usepackage{stix}

\DeclarePairedDelimiter\abs{\lvert}{\rvert}%
\DeclarePairedDelimiter\norm{\lVert}{\rVert}%

\makeatletter
\let\oldabs\abs
\def\abs{\@ifstar{\oldabs}{\oldabs*}}
%
\let\oldnorm\norm
\def\norm{\@ifstar{\oldnorm}{\oldnorm*}}
\makeatother

\newcommand{\N}{\mathbb{N}}
\newcommand{\Z}{\mathbb{Z}}

% \delimitershortfall=0pt
\renewcommand\arraystretch{2}

% \newtheorem{theorem}{Theorem}
\numberwithin{equation}{section}

\newenvironment{theorem}[2][Theorem]{\begin{trivlist}
\item[\hskip \labelsep {\bfseries #1}\hskip \labelsep {\bfseries #2.}]}{\end{trivlist}}
\newenvironment{lemma}[2][Lemma]{\begin{trivlist}
\item[\hskip \labelsep {\bfseries #1}\hskip \labelsep {\bfseries #2.}]}{\end{trivlist}}
\newenvironment{exercise}[2][Exercise]{\begin{trivlist}
\item[\hskip \labelsep {\bfseries #1}\hskip \labelsep {\bfseries #2.}]}{\end{trivlist}}
\newenvironment{reflection}[2][Reflection]{\begin{trivlist}
\item[\hskip \labelsep {\bfseries #1}\hskip \labelsep {\bfseries #2.}]}{\end{trivlist}}
\newenvironment{proposition}[2][Proposition]{\begin{trivlist}
\item[\hskip \labelsep {\bfseries #1}\hskip \labelsep {\bfseries #2.}]}{\end{trivlist}}
\newenvironment{corollary}[2][Corollary]{\begin{trivlist}
\item[\hskip \labelsep {\bfseries #1}\hskip \labelsep {\bfseries #2.}]}{\end{trivlist}}

\begin{document}

% --------------------------------------------------------------
% DOCUMENT
% --------------------------------------------------------------

\title{Supplemental Note: Dynamical model of Cas13d gRNA excision for negative-autoregulatory optimization}
\author{Chase P. Kelley, Eric T. Wang\\
University of Florida}

\maketitle

\section{Model description}

\noindent The following model is proposed to describe the kinetics and equilibria of a Cas13d negative autoregulation strategy mediated by gRNA excision (GENO):
\begin{align*}
  \dot{R} &= \text{transcription} - \text{processing} - \text{degradation}\\
  \dot{A} &= \text{translation} - \text{processing} - \text{degradation}\\
  \dot{B} &= \text{processing} - \text{degradation}
\end{align*}

\noindent This description relies on the following assumptions:
\begin{enumerate}
  \item A deterministic model describes the mean dynamics of the underlying stochastic process with reasonable accuracy.
  \item gRNA processing is performed by apoprotein only, upon Cas13d:gRNA binary complex formation.
  \item Nascent Cas13d translation products quickly diffuse away from the domain of their mRNAs.
  \item Cas13d:gRNA binary complex formation is irreversible.
  \item Nuclear/cytoplasmic compartmentalization affects crRNA processing negligibly.
\end{enumerate}

\noindent These biochemical dynamics and assumptions produce the following differential equation model (GENO):
\begin{align}
  \dot{R} &= r_t - k_p R A - \gamma_R R\\
  \dot{A} &= k_T R - k_p R A - \gamma_A A\\
  \dot{B} &= k_p R A - \gamma_B B
\end{align}

\noindent where:
\begin{itemize}[itemsep=0pt]
  \item $R$: concentration of Cas13d mRNA
  \item $A$: concentration of Cas13d apoprotein
  \item $B$: concentration of Cas13d:gRNA binary complex
  \item $r_t$: rate of Pol II transcription of Cas13d mRNA
  \item $k_T$: rate of translation of Cas13d protein
  \item $k_p$: rate of crRNA processing
  \item $\gamma_i$: rate of degradation of species $i$
\end{itemize}

\vspace*{2mm}

\noindent A reference model (REF) is also presented for comparison, in which autoregulation is absent and gRNA processing and Cas13d expression are independent:
\begin{align}
  \dot{R} &= r_t - \gamma_R R\\
  \dot{A} &= k_T R - k_p G A - \gamma_A A\\
  \dot{B} &= k_p G A - \gamma_B B\\
  \dot{G} &= r_G - k_p G A - \gamma_G G
\end{align}

\noindent where:
\begin{itemize}[itemsep=0pt]
  \item $G$: concentration of unbound gRNA
  \item $r_G$: rate of Pol III transcription of gRNA
\end{itemize}

\section{Equilibrium analysis}

\subsection{GENO model}

\noindent At equilibrium, the GENO model reduces to the following:
\begin{align}
  0 &= r_t - k_p \hat{R} \hat{A} - \gamma_R \hat{R} \label{GEeq1}\\
  0 &= k_T \hat{R} - k_p \hat{R} \hat{A} - \gamma_A \hat{A} \label{GEeq2}\\
  0 &= k_p \hat{R} \hat{A} - \gamma_B \hat{B} \label{GEeq3}
\end{align}

\noindent From (\ref{GEeq1}),
\begin{align}
  \hat{A} &= \frac{r_t - \gamma_R \hat{R}}{k_p \hat{R}} \label{GEeq4}
\end{align}

\noindent By substitution of (\ref{GEeq4}) into (\ref{GEeq2}),
\begin{align}
  \frac{r_t - \gamma_R \hat{R}}{k_p \hat{R}} &= \frac{k_T \hat{R}}{k_p \hat{R} + \gamma_A}
\end{align}

\noindent which reduces to the quadratic equation
\begin{align}
  0 &= k_p (k_T + \gamma_R) \hat{R}^2 - (k_p r_t - \gamma_A \gamma_R) \hat{R} - \gamma_A r_t \label{GEeq5}
\end{align}

\noindent The solutions of this quadratic equation are
\begin{align}
  \hat{R} &= \frac{k_p r_t - \gamma_A \gamma_R \pm \sqrt{(k_p r_t - \gamma_A \gamma_R)^2 + 4 \gamma_A r_t k_p (k_T + \gamma_R)}}{2 k_p (k_T + \gamma_R)}
\end{align}

\noindent By inspection,
\begin{align}
  \abs{k_p r_t - \gamma_A \gamma_R} &< \sqrt{(k_p r_t - \gamma_A \gamma_R)^2 + 4 \gamma_A r_t k_p (k_T + \gamma_R)}
\end{align}

\noindent Therefore, regardless of the sign of the term $\abs{k_p r_t - \gamma_A \gamma_R}$, the quadratic equation (\ref{GEeq5}) has a single positive solution given by
\begin{align}
  \hat{R} &= \frac{k_p r_t - \gamma_A \gamma_R + \sqrt{(k_p r_t - \gamma_A \gamma_R)^2 + 4 \gamma_A r_t k_p (k_T + \gamma_R)}}{2 k_p (k_T + \gamma_R)}
\end{align}

\noindent Equivalently,
\begin{align}
  \hat{R} &= \dfrac{r_t}{\gamma_R} \left( \dfrac{k_p \gamma_R - \dfrac{\gamma_A \gamma_R^2}{r_t} + \sqrt{\left( k_p \gamma_R - \dfrac{\gamma_A \gamma_R^2}{r_t} \right)^2 + \dfrac{4 k_p k_T \gamma_A \gamma_R^2}{r_t}} }{2 k_p (k_T + \gamma_R)} \right) \label{GEeq7}
\end{align}

\noindent From (\ref{GEeq3}) and (\ref{GEeq4}),
\begin{align}
  \hat{B} &= \frac{r_t - \gamma_R \hat{R}}{\gamma_B} \label{GEeq6}
\end{align}

\noindent Thus, the system described by the GENO model has a single equilibrium point $(\hat{R}^{\mathrm{GENO}}, \hat{A}^{\mathrm{GENO}}, \hat{B}^{\mathrm{GENO}})$ at the solutions provided in (\ref{GEeq7}), (\ref{GEeq4}), and (\ref{GEeq6}).

\subsection{REF model}

\noindent At equilibrium, the REF model reduces to the following:
\begin{align}
  0 &= r_t - \gamma_R \hat{R} \label{REFeq1}\\
  0 &= k_T \hat{R} - k_p \hat{G} \hat{A} - \gamma_A \hat{A} \label{REFeq2}\\
  0 &= k_p \hat{G} \hat{A} - \gamma_B \hat{B} \label{REFeq3}\\
  0 &= r_G - k_p \hat{G} \hat{A} - \gamma_G \hat{G} \label{REFeq4}
\end{align}

\noindent From (\ref{REFeq1}),
\begin{align}
  \hat{R} &= \frac{r_t}{\gamma_R} \label{REFeq5}
\end{align}

\noindent From (\ref{REFeq2}) and (\ref{REFeq5}),
\begin{align}
  \hat{A} &= \frac{k_T r_t}{\gamma_R} \left( \frac{1}{k_p \hat{G} + \gamma_A} \right) \label{REFeq6}
\end{align}

\noindent From (\ref{REFeq4}),
\begin{align}
  \hat{G} &= \frac{r_G}{k_p \hat{A} + \gamma_G} \label{REFeq7}
\end{align}

\noindent Substituting (\ref{REFeq7}) into (\ref{REFeq6}) yields
\begin{align}
  \hat{A} &= \dfrac{k_T r_t}{\gamma_R} \left( \dfrac{1}{\dfrac{k_p r_G}{k_p \hat{A} + \gamma_G} + \gamma_A} \right)
\end{align}

\noindent which simplifies to the quadratic equation
\begin{align}
  \gamma_A k_p \hat{A}^2 + \left( k_p r_G + \gamma_A \gamma_G - \frac{k_T r_t k_p}{\gamma_R} \right) \hat{A} - \frac{k_T r_t \gamma_G}{\gamma_R} &= 0 \label{REFeq9}
\end{align}

\noindent As the product of the first and third coefficients is strictly negative, the discriminant is strictly positive. Thus, the polynomial has a single positive real solution for $\hat{A}$.\\

\noindent From (\ref{REFeq3}),
\begin{align}
  \hat{B} &= \frac{k_p \hat{G} \hat{A}}{\gamma_B} \label{REFeq8}
\end{align}

\noindent Substituting (\ref{REFeq6}) into (\ref{REFeq8}) yields
\begin{align}
  \hat{B} &= \frac{k_T r_t}{\gamma_R \gamma_B} \left( \frac{k_p \hat{G}}{k_p \hat{G} + \gamma_A} \right) \label{REFeq10}
\end{align}

\noindent Thus, the system described by the REF model has a single equilibrium point\\ $(\hat{R}^{\mathrm{REF}}, \hat{A}^{\mathrm{REF}}, \hat{B}^{\mathrm{REF}}, \hat{G}^{\mathrm{REF}})$ with the solution fully constrained by (\ref{REFeq5}), (\ref{REFeq9}), (\ref{REFeq10}), and (\ref{REFeq7}), respectively. \\

\noindent Importantly, in the REF model, the concentration of binary complex $\hat{B}$ takes the form of a Hill function, where $\lim\limits_{\hat{G} \to 0} \hat{B} = 0$ and $\lim\limits_{\hat{G} \to \infty} \hat{B} = \frac{k_T r_t}{\gamma_R \gamma_B}$. \\

\noindent The autoregulation efficiency $\eta_{\mathrm{GENO}}$ is defined as
\begin{align}
  \eta_{\mathrm{GENO}} &= \frac{\hat{B}^{\mathrm{REF}} - \hat{B}^{\mathrm{GENO}}}{\hat{B}^{\mathrm{REF}}}
\end{align}

\section{Proofs}

\begin{theorem}{1}
  At equilibrium, negative autoregulation by gRNA excision reduces the expression of the Cas13d mRNA compared to the reference model.
\end{theorem}

\begin{proof}
  Proof by contradiction. The equilibrium mRNA concentration in the GENO model is provided in (\ref{GEeq7}):
  \begin{align*}
    \hat{R}^{\mathrm{GENO}} &= \dfrac{r_t}{\gamma_R} \left( \dfrac{k_p \gamma_R - \dfrac{\gamma_A \gamma_R^2}{r_t} + \sqrt{\left( k_p \gamma_R - \dfrac{\gamma_A \gamma_R^2}{r_t} \right)^2 + \dfrac{4 k_p k_T \gamma_A \gamma_R^2}{r_t}} }{2 k_p (k_T + \gamma_R)} \right)
  \end{align*}

  \noindent In the REF model, the equilibrium concentration is provided in (\ref{REFeq5}):
  \begin{align*}
    \hat{R}^{\mathrm{REF}} &= \frac{r_t}{\gamma_R}
  \end{align*}

  \noindent Assume $\hat{R}^{\mathrm{GENO}} \geq \hat{R}^{\mathrm{REF}}$. Equivalently,
  \begin{align*}
    \dfrac{k_p \gamma_R - \dfrac{\gamma_A \gamma_R^2}{r_t} + \sqrt{\left( k_p \gamma_R - \dfrac{\gamma_A \gamma_R^2}{r_t} \right)^2 + \dfrac{4 k_p k_T \gamma_A \gamma_R^2}{r_t}} }{2 k_p (k_T + \gamma_R)} &\geq 1\\
    k_p \gamma_R - \dfrac{\gamma_A \gamma_R^2}{r_t} + \sqrt{\left( k_p \gamma_R - \dfrac{\gamma_A \gamma_R^2}{r_t} \right)^2 + \dfrac{4 k_p k_T \gamma_A \gamma_R^2}{r_t}} &\geq 2 k_p k_T + 2 k_p \gamma_R\\
    \sqrt{\left( k_p \gamma_R - \dfrac{\gamma_A \gamma_R^2}{r_t} \right)^2 + \dfrac{4 k_p k_T \gamma_A \gamma_R^2}{r_t}} &\geq 2 k_p k_T + k_p \gamma_R + \dfrac{\gamma_A \gamma_R^2}{r_t}\\
    \left( k_p \gamma_R - \dfrac{\gamma_A \gamma_R^2}{r_t} \right)^2 + \dfrac{4 k_p k_T \gamma_A \gamma_R^2}{r_t} &\geq \left( 2 k_p k_T + k_p \gamma_R + \dfrac{\gamma_A \gamma_R^2}{r_t} \right)^2\\
  \end{align*}
  \vspace*{-15mm}
  \begin{align*}
    \left( \dfrac{\gamma_A \gamma_R^2}{r_t} \right)^2 + \dfrac{2 \gamma_A \gamma_R^3 k_p}{r_t} + k_p^2 \gamma_R^2 + \dfrac{4 k_p k_T \gamma_A \gamma_R^2}{r_t} &\geq \left( \dfrac{\gamma_A \gamma_R^2}{r_t} \right)^2 + \dfrac{2 \gamma_A \gamma_R^3 k_p}{r_t} + \\
    &k_p^2 \gamma_R^2 + \dfrac{4 k_p k_T \gamma_A \gamma_R^2}{r_t} + 4 k_p^2 k_T \gamma_R + 4 k_p^2 k_T^2\\
    0 &\geq 4 k_p^2 k_T \gamma_R + 4 k_p^2 k_T^2\\
    0 &\geq 4 k_p^2 k_T (\gamma_R + k_T)
  \end{align*}

  \noindent This inequality cannot be satisfied, as all biological parameters in the right-hand term are strictly positive. Therefore, $\hat{R}^{\mathrm{GENO}} < \hat{R}^{\mathrm{REF}}$.

\end{proof}

\begin{theorem}{2}
  Assume that gRNA is highly expressed in the reference model and is present in excess $(\hat{G} >> \gamma_A/k_p)$, and that Cas13d protein translation is faster than mRNA degradation $(k_T > \gamma_R)$. At equilibrium, negative autoregulation by gRNA excision reduces the concentration of active Cas13d:gRNA binary complex compared to the reference model.
\end{theorem}

\begin{proof}
  Proof by contradiction. The equilibrium binary complex concentration in the GENO model is provided in (\ref{GEeq6}):
  \begin{align*}
    \hat{B}^{\mathrm{GENO}} &= \frac{r_t - \gamma_R \hat{R}^{\mathrm{GENO}}}{\gamma_B}
  \end{align*}

  \noindent In the REF model, the equilibrium concentration is provided in (\ref{REFeq10}):
  \begin{align*}
    \hat{B}^{\mathrm{REF}} &= \frac{k_T r_t}{\gamma_R \gamma_B} \left( \frac{k_p \hat{G}^{\mathrm{REF}}}{k_p \hat{G}^{\mathrm{REF}} + \gamma_A} \right)
  \end{align*}

  \noindent If gRNA is expressed in excess in the REF model, the binary complex equilibrium concentration approaches its maximum value:
  \begin{align*}
    \lim_{\hat{G} \to \infty} \hat{B}^{\mathrm{REF}} &= \frac{k_T r_t}{\gamma_R \gamma_B}
  \end{align*}

  \noindent Assume $\hat{B}^{\mathrm{GENO}} \geq \lim\limits_{\hat{G} \to \infty} \hat{B}^{\mathrm{REF}}$. Equivalently,
  \begin{align*}
    \frac{r_t - \gamma_R \hat{R}^{\mathrm{GENO}}}{\gamma_B} &\geq \frac{k_T r_t}{\gamma_R \gamma_B}\\
    r_t - \gamma_R \hat{R}^{\mathrm{GENO}} &\geq \frac{k_T r_t}{\gamma_R}\\
    1 - \frac{k_T}{\gamma_R} &\geq \frac{\gamma_R \hat{R}^{\mathrm{GENO}}}{r_t}
  \end{align*}

  \noindent The right-hand term of this inequality is strictly positive. However, as $k_T > \gamma_R$, the left-hand term is strictly negative. This inequality cannot be satisfied. Thus, $\hat{B}^{\mathrm{GENO}} < \lim\limits_{\hat{G} \to \infty} \hat{B}^{\mathrm{REF}}$, under the following conditions:
  \begin{enumerate}
    \item gRNA expression in the reference system is high and in excess.
    \item On average, more than one Cas13d protein molecule is translated from each Cas13d mRNA in the reference system.
  \end{enumerate}

\end{proof}

\end{document}
